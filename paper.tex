\documentclass[11pt]{article}

\usepackage[utf8]{inputenc}
\usepackage[czech]{babel}
\usepackage{graphicx}
%\usepackage{times}
\usepackage{multirow}

\title{Statistické metody \\ Domací úkol}

\author{Matouš Macháček \\
  {\tt machacekmatous@gmail.com} }

\date{}

\begin{document}
\maketitle

\section{Data}

Pro účely domácího úkolu byly dodány dva soubory: Pro češtinu to byl soubor \texttt{TEXT1CZ.txt},
který obsahoval $222412$ tokenů. Pro angličtinu byl použit soubor \texttt{TEXT1EN.txt}, který
obsahoval $221098$ tokenů. Různé 
statistiky spočítané z těchto souborů se nacházejí v tabulce \ref{data}.

\begin{figure}
    \begin{center}
    \begin{tabular}{|l|c|c|}
        \hline
        & Czech & English \\
        \hline
        \input{results-stats.table}
        \hline
    \end{tabular}
    \end{center}
    \caption{Comparing Czech and English language}
    \label{data}
\end{figure}

Pro další srovnání textů jsme spočítali, kolik se vyskytuje slov s danou četností. Výsledky
jsou zanesené do grafu \ref{zipf}.

\begin{figure}
    \includegraphics{results-zipf.pdf}
    \caption{Count of words with a given frequency}
    \label{zipf}
\end{figure}

\section{Entropie textu}

V tomto experimentu měříme podmíněnou entropii v textu pro dané předchozí slovo. To je počítáno pomocí skriptu 
\texttt{entropy-experiment.pl} a modulu \texttt{LanguageEntropy.pm}. Výsledky jsou v tabulce \ref{entropy-basic-results}

\begin{figure}
    \begin{center}
    \begin{tabular}{|l|c|c|}
        \hline
        File & Entropy & Perplexity \\
        \hline
        \texttt{TEXT1CZ.txt} & 4.748 & 26.871 \\
        \texttt{TEXT1EN.txt} & 5.287 & 39.043 \\
        \hline
    \end{tabular}
    \end{center}
    \caption{Conditional entropy basic results }
    \label{entropy-basic-results}
\end{figure}

\subsection{Náhodná změna slov}

V další části experimentu jsme každé slovo v textu s určitou pravděpodobností změnili za libovolné jiné slovo. 
Pak jsme sledovali, jak se mění podmíněná entropie v závislosti na pravděpodobnosti, se kterou jsme měnili slova.
V tabulce \ref{results-entropy-words} se nacházejí výsledky z tohoto experimentu. Pro kažkou hodnotu pravděpodobnosti,
jsme entropii změřili desetkrát. V tabulce je tedy minimální, průměrná a maximální entropie z těchto deseti měření.

V grafech \ref{results-entropy-CZ} a \ref{results-entropy-EN} jsou tato data znázorněna červenými body (zeleně jsou 
znázorněny data z dalšího experimentu).

\begin{figure}
    \begin{center}
    \begin{tabular}{|l|ccc|ccc|}
        \hline
               & \multicolumn{3}{|c|}{Czech} & \multicolumn{3}{|c|}{English} \\
        Rate of changed words & min & avg & max & min & avg & max \\
        \hline
        \input{results-entropy-words.table}
        \hline
    \end{tabular}
    \end{center}
    \caption{Entropy of the texts, where the words were changed with the given probabilities}
    \label{results-entropy-words}
\end{figure}


\begin{figure}
    \includegraphics{results-entropy-CZ.pdf}
    \caption{Entropy of the Czech text, where the items were changed with the given probabilities}
    \label{results-entropy-CZ}
\end{figure}

\begin{figure}
    \includegraphics{results-entropy-EN.pdf}
    \caption{Entropy of the English text, where the items were changed with the given probabilities}
    \label{results-entropy-EN}
\end{figure}

Z grafů plyne, že pokud zvyšujeme pravděpodobnost záměny slov, bude se entropie zvyšovat pro angličtinu,
ale pro češtinu se bude snižovat. Myslím, že by se to dalo vysvětlit grafem \ref{zipf}. V češtině se slovo zamění většinou
na nějaké málo frekventované. Naopak v angličtině se slovo zamění s větší pravděpodobností na nějake častější slovo. Tím 
se distribuce pravděpodobností stává více uniformní a entropie se zvyšuje.

\subsection{Náhodná změna znaků}

Podobně, jako jsme měnili slova s určitou pravděpodobností, tak jsme také náhodně měnili jednotlivé znaky. 
Výsledky z toho experimentu se nacházejí v tabulce \ref{results-entropy-chars}. Vizuálně jsou tyto výsledky
znázorněny zeleně v grafech \ref{results-entropy-CZ} a \ref{results-entropy-EN} (tentokrát rozděleno podle
jazyka).

\begin{figure}
    \begin{center}
    \begin{tabular}{|l|ccc|ccc|}
        \hline
               & \multicolumn{3}{|c|}{Czech} & \multicolumn{3}{|c|}{English} \\
        Rate of changed chars & min & avg & max & min & avg & max \\
        \hline
        \input{results-entropy-chars.table}
        \hline
    \end{tabular}
    \end{center}
    \caption{Entropy of the texts, where the characters were changed with the given probabilities}
    \label{results-entropy-chars}
\end{figure}

U tohoto experimentu klesají entropie u obou jazyků. Na rozdíl od předchozího experimentu, tady
dochází k záměně na úplně nová slova. Zvyšuje se velikost slovníku a přitom délka textu zůstává stejná.
Je spousta slov, která jsou viděna pouze jednou a která jednoznačně určují následující slovo. To snižuje
entropii.

Poznamenejme ještě, že pokud bychom zaměnili každý znak, byla by entropie nejnižší. To však platí pouze
pro danou délku textu. Pokud bychom text prodlužovali, entropie by se postupně zvyšovala (protože by
se začala opakovat slovat a už by neurčovala jednoznačně svého následníka).

\subsection{Teoretické cvičení}

Nechť množiny $I_1$, $J_1$, $I_2$, $J_2$ jsou množiny z definice entropie textů $T_1$ a $T_2$.
Můžeme pak množiny $I$ a $J$ vyjádřit jako jejich disjunktní sjednocení:

\begin{equation}
    I = I_1 \cup I_2
\end{equation}

\begin{equation}
    J = J_1 \cup J_2
\end{equation}

Výpočet podmíněné entropie podle definice
\begin{equation}
H(J|I) =-\sum_{i \in I, j \in J} P(i,j) log_2 P(j|i)
\end{equation}

Můžeme přepsat takto:
\begin{equation}
H(J|I) = K(J_1|I_1) + K(J_2|I_2) + K(J_1|I_2) + K(J_2|I_1)
\label{rovK}
\end{equation}

kde
\begin{equation}
    K(A|B) = -\sum_{i \in B, j \in A} P(i,j) log_2 P(j|i)
\end{equation}

%Pro $i \in I_1$ a $j \in J_2$ neexistuje žádný bigram $(i,j)$
%a tedy $P(i,j)=0$. To se však vyskytuje v sumě $K(J_1|I_2)$
%a tedy $K(J_1|I_2) = 0$. 
%Nejdříve ukážeme, že $K(J_1|I_2)=K(J_2|I_1)=0$. Pro každou

Dále pro $i \in I_1$ a $j \in J_1$ platí:
\begin{equation}
    P(i,j) = \frac{C(i,j)}{2L} = \frac{1}{2} \frac{C(i,j)}{L} = \frac{1}{2} P_1(i,j)
\end{equation}

To a zřejmý fakt, že $P(j|i) = P_1(j|i)$ použijeme v následujícím vyjádření:
\begin{eqnarray}
    K(J_1|I_1) & =& -\sum_{i \in I_1, j \in J_1} P(i,j) log_2 P(j|i) \\
    & = & -\sum_{i \in I_1, j \in J_1} \frac{1}{2} P_1(i,j) log_2 P_1(j|i) \\
    & = & \frac{1}{2} E
\end{eqnarray}

Podobně bychom dospěli k tomuto:
\begin{equation}
    K(J_1|I_1) = \frac{1}{2} E
\end{equation}

Dosazením do \ref{rovK} pak dostáváme:
\begin{equation}
H(J|I) = \frac{1}{2} E + \frac{1}{2} E + K(J_1|I_2) + K(J_2|I_1) = E + K(J_1|I_2) + K(J_2|I_1)
\end{equation}

Víme, že oba sčítance $K(J_1|I_2)$ a $K(J_2|I_1)$ jsou nezáporné, protože je nezáporný každý sčítanec sumy.
Tím jsme tedy dokázali, že
\begin{equation}
    H(J|I) \geq E
\end{equation}

\section{Jazykový model}

Úkolem druhé části domácího úkolu bylo natrénovat trigramový jazykový model. K tomu jsme použili skript 
\texttt{language-model-experiment.pl} a model \texttt{LanguageModel.pm}. 

Oba texty byly rozděleny takto: posledních 20 000 tokenů bylo z každého souboru odstraněno abychom z nich mohli
vytvořit testset. Potom bylo odebráno dalších 40 000 tokenů z konce jako heldout data. Zbytek 
jsme použili jako trainset.

Pomocí trainsetu jsme natrénovali (metodou MLE) unigramový, bigramový a trigramový model. Dále jsme pomocí heldout
setu natrénovali (metodou EM) lambda koeficienty. Tím jsme získali interpolovaný a vyhlazený trigramový model.
Na testsetu jsme dosáhli hodnoty Cross-entropy $10.221$ pro češtinu a $7.468$ pro angličtinu.

Zkoušeli jsme také trénovat lambdy na trainsetu. Podle předpokladu šel koeficient trigramového modelu rychle k jedničce a ostatní
koeficienty naopak klesali rychle k nule. Log z tohoto trénování se nachází v souborech \texttt{languagemodel.CZ.log} a 
\texttt{languagemodel.EN.log}. 

\subsection{Ladění koeficientu trigramového modelu}

V další části experimentu jsme zkoušeli měnit postupně parametr $\lambda_3$ a dívali jsme se, jak se mění Cross-Entropy na testsetu. 
Tyto výsledky jsou znázorněny v grafu \ref{results-model} a v tabulkách \ref{results-model-CZ} a \ref{results-model-EN}.

\begin{figure}
    \includegraphics{results-model.pdf}
    \caption{Graph of entropies for English text.}
    \label{results-model}
\end{figure}

\begin{figure}
    \begin{center}
    \begin{tabular}{|l|c|}
        \hline
        $\lambda_3$ & Cross-Entropy \\
        \hline
        \input{results-model-CZ.table}
        \hline
    \end{tabular}
    \end{center}
    \caption{Cross-Entropy of Czech language model computed on testset with the given lambdas}
    \label{results-model-CZ}
\end{figure}

\begin{figure}
    \begin{center}
    \begin{tabular}{|l|c|}
        \hline
        $\lambda_3$ & Cross-Entropy \\
        \hline
        \input{results-model-EN.table}
        \hline
    \end{tabular}
    \end{center}
    \caption{Cross-Entropy of English language model computed on testset with the given lambdas}
    \label{results-model-EN}
\end{figure}

Z grafu je vidět, že EM algoritmus určil optimální hodnotu $\lambda_3$ správně. Tato hodnota je pro oba jazyky podobná. 
Když se parametr blíží k jedničce, Cross-Entropy rychle stoupá. To by se dalo vysvětlit tím, že trénovací množina je relativně
malá a trigramový model je tím pádem málo robustní. Na tomto je vidět, jak důležité vyhlazování jazykového modelu.

Pokud naopak nastavíme parametr na nulu (a tedy trigramový model
vůbec nevyužíváme), Cross-Entropy se zvýší jen mírně. Dalo by se to tedy říct i tak, že na daných trénovacích datech si trigramovým
modelem oproti bigramovému moc nepomůžeme. Dalo by se předpokládat, že čtyřgramovým modelem bychom si pomohli ješte méně.

Nakonec je v tomto grafu vidět, že pro češtinu jsou hodnoty Cross-Entropy znatelně vyšší. To by se dalo vysvětlit tím, že v českém 
textu se vyskytuje přibližně čtyřikrát více různých slov a český model má tedy daleko více parametrů. Pří podobné velikosti trénovacích
dat pak musí být český model méně robustní. Další možné vysvětlení je, že v češtině je volný slovosled a narozdíl od angličtiny je tedy
težší předpovídat následující slova.

\end{document}
